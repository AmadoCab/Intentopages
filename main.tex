\documentclass{article}
\usepackage[utf8]{inputenc}
\usepackage{hyperref}

\setlength{\parindent}{0pt}

\title{Open Souce por Guatemala}
\author{Amado Alberto Cabrera Estrada\\
201905757}
\date{4 de noviembre 2020}

\begin{document}

\maketitle

%Introducción
En el mundo de la programación uno de los conceptos más extendidos –y vividos– por la comunidad, es el concepto del \textit{Open Source}\footnote{ El software open source es un código diseñado de manera que sea accesible al público: todos pueden ver, modificar y distribuir el código de la forma que consideren conveniente. El software open source se desarrolla de manera descentralizada y colaborativa, así que depende de la revisión entre compañeros y la producción de la comunidad.\\ –\href{https://www.redhat.com/es/topics/open-source/what-is-open-source}{RedHat}} (Código abierto). El \textit{Open Source} consiste en permitir que un producto diseñado por una empresa o individuo sea libremente usado, distribuido o modificado por cualquier persona. Pero ¿Qué relación tiene el desarrollo de un país con el \textit{software open source}?\\

%Desarrollo
El \textit{Open Source} une a la comunidad. Uno de los grandes beneficios del \textit{Open Source} es que este propicia la unión de la comunidad participe de el proyecto y el crecimiento de este. En este artículo usaremos el término de modo más amplio refiriéndonos, no solo al software sino, a la serie de valores que propicia. Como:
\begin{itemize}
    \item \textbf{Retroalimentación:}
    Dado que cualquier persona puede acceder al código fuente de manera libre las comunidades que se benefician del \textit{software} trabajan por mantenerlo, mejorarlo y distribuirlo. Esto fomenta el crecimiento para todos los usuarios.
    
    \item \textbf{Transparencia:}
    En esta época donde los datos se volvieron el más preciado de los bienes por toda clase de empresas sin escrúpulos el \textit{software open source} permite que cada individuo –con el conocimiento adecuado– conozca el funcionamiento interno del servicio que usa, sin depender de las promesas de los proveedores.
    
    \item \textbf{Menor costo:}
    Pese a que el código es en sí gratuito el valor del proyecto aumenta cuando una comunidad lo mejora, además de darle soporte, actualizarlo y distribuirlo aun más de forma completamnte gratuita (además de muchas veces colaborar con donativos).
\end{itemize}
Estos valores hacen que un proyecto \textit{open source} sea más viable y beneficioso para cada una de las partes. Sin embargo no solo haremos promesas de la viabilidad de estos proyectos.\\

Es normal dudar de la efectividad de estos proyectos –pese a lo bien que suenen ``en papel"– quiero decir, es difícil de creer que al regalar tu proyecto a la comunidad obtengas algún beneficio, sin embargo mostraré que es al contrario. El ejemplo más paradigmático de esto es el sistema operativo \textit{Linux} el cual cuenta con una de las cuotas más grandes del mercado\footnote{\href{https://en.wikipedia.org/wiki/Usage_share_of_operating_systems\#Market_share_by_category}{Datos \textit{Linux}}} en cuanto a sistemas operativos. Es incluso muy difícil referirse a \textit{Linux} sin ponerle ningún apellido ya que lo que hoy día llamamos \textit{Linux} es realmente un conjunto inmenso de variaciones del mismo\footnote{\href{https://upload.wikimedia.org/wikipedia/commons/1/1b/Linux_Distribution_Timeline.svg}{Distribuciones del \textit{kernel Linux}}} cosa que es sí misma ya demuestra el éxito que ha tenido este sistema operativo creado en 1991 por Linus Torvalds\footnote{Aquí se habla del \textit{kernel} del sistema operativo que es el que ha sido usado de manera más extendida cambiando unicamente capas superiores de la construcción}. Si hablamos de \textit{RedHat Enterprise} (un sistema operativo basado en \textit{Linux}) que ha creado una empresa completa dedicada al \textit{open source} siendo líder en proveer servicios a grandes empresas. Sin embargo esto no tiene por que limitarse a la programación.\\

Los proyectos \textit{open source} se abren paso por el mundo en proyectos en los cuales nunca se los imaginaria. Este es el caso del proyecto \textit{Farming for the future}\footnote{\href{https://www.redhat.com/es/open-source-stories/farming-for-the-future}{Articulo de \textit{Farming for the future}}}. \textit{Farming for the future} es un proyecto que nace por el proyecto de investigación \textit{MARSfarm} el cual consiste en encontrar el método más idóneo para hacer crecer distintas especies de plantas, de modo que con condiciones mínimas se pueda cultivar alimento en Marte. Este proyecto se extendió a escuelas en las cuales estudiantes desde 6to primaria se han envuelto en el desarrollo de computadoras para el cultivo. Este esfuerzo conjunto permite que los investigadores tengan un equipo de trabajo que no depende solo de ellos o la financiación que se les brinde, ya que estas soluciones se han elaborado en centros educativos alrededor de Estados Unidos. Y esto no es todo pues abaratando costos de sensores especializados para el crecimiento de las plantas (debido a las soluciones que se han encontrado) facilitan que granjeros utilicen estas tecnologías en sus propios campos llegando incluso a cultivar en lugares en los que antes no se podía.\\

%Conclusión
Por todas las razones arriba comentadas es que creo que el desarrollo se encuentra en el \textit{open source}, en la creación de proyectos de esta naturaleza. En resumen crear proyectos \textit{open source} permite que todos avancemos y nos une como comunidad reforzando confianza y cooperatividad, entre otros. La utilidad de los proyectos de este tipo es tangible en muchos proyectos ya incluso fuera del mundo especializado\footnote{\href{https://www.redhat.com/en/open-source-stories}{Más proyectos}}. Es justamente esto lo que me lleva a instar a mis lectores a crear uno de estos proyectos o colaborar en estos para que extendamos el uso de estos métodos. Incluso el practicar los valores del \textit{open source} en nuestro día a día nos ayuda a todos. Creo que el tiempo está –y seguirá– demostrando que esta es la mejor manera de proceder, con ejemplos como \textit{\href{https://reactjs.org/}{React.js}} (de Facebook) o \textit{\href{https://vuejs.org/}{Vue.js}} (de un particular).\\

Además quería instar a profesores a pensar como puede ayudarnos la libertad de distribución en la educación. Particularmente para la ECFM he pensado en un curso \textit{online} disponible en la página de la escuela para preparar alumnos para el examen de ingreso pues de ese modo sin bajar los estándares de admisión habrá más posibilidad de conseguir alumnos cosa que nos ayudará a todos.\\

%Conclusión de la conclusión
Es claro que al hablar de progreso siempre se colocan muchas cartas en el tablero y hay que evaluar cuál es la más óptima para cada caso –ni mi idea está exenta de ello–, pero con este pequeño artículo espero que hayas visto, lector mio, mi idea para alcanzar el desarrollo. Y desde luego espero haberte animado a ser participe en esta idea.\\

Fiel a las ideas expresadas en mi articulo dejo un link al archivo original escrito en \LaTeX\footnote{\href{https://www.overleaf.com/docs?snip_uri=https://amadocab.github.io/Intentopages/main.tex}{Código fuente en Overleaf}}\footnote{\href{https://amadocab.github.io/Intentopages/main.tex}{Descargar código fuente}}

%revisar sino's
%comillas

\end{document}
